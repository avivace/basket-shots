\documentclass[12pt,a4paper]{report}
\usepackage{url}
\usepackage{hyperref}
\usepackage{pdfpages}
\usepackage{titlesec, blindtext, color}
\usepackage{amsmath}
\usepackage{amsfonts}
\usepackage{graphicx}
\usepackage{relsize} 
\usepackage{wrapfig}
\usepackage{setspace}
\usepackage{amsthm}
%\usepackage[utf8]{inputenc}
% Allows setting fixed width in tables columns
\usepackage{array}
% Allows breaking tables in multiple pages
\usepackage{longtable}
% Syntax highlighted code blocks using Pygment
\usepackage{minted}
% Use custom fonts, used to redefine the monospaced font
\usepackage{fontspec}
% Colors
\usepackage{xcolor}

\setmonofont{Iosevka}[Scale=0.85]

% Code blocks labels
\newenvironment{code}{\captionsetup{type=listing}}{}

\usepackage[font=footnotesize, labelfont=bf]{caption}
\captionsetup[table]{position=below}
\captionsetup[figure]{position=below}

% Chapter titles
\graphicspath{ {images/} }
\definecolor{gray90}{gray}{0.90}
\titleformat{\chapter}[hang]{\Huge\bfseries}{\thechapter\hsp\textcolor{gray90}{|}\hsp}{0pt}{\Huge\bfseries}
\newcommand{\hsp}{\hspace{10pt}}

% Redefine Table of Contents title
\renewcommand{\contentsname}{Indice}

\begin{document}

\title{%
  \Huge Progetto Basket Shots\\
  \large Predizione dell'esito di tiri nella pallacanestro competitiva utilizzando Support Vector Machines\\
    }
\author{
  Coppola Matteo\\
  \texttt{793329}
  \and
  Palazzi Luca\\
  \texttt{793556}
   \and
  Vivace Antonio\\
  \texttt{793509}
}
\date{Data Technology e Machine Learning, \\ Febbraio 2019}
\maketitle

\tableofcontents

\chapter{Introduzione}

\section{Dominio di riferimento e obbiettivi}
Il settore sportivo professionistico è in costante ricerca di strumenti, modelli e modalità che permettano di analizzare gli andamenti delle partite e le performance dei singoli giocatori.

Il basket americano, ad esempio, utilizza in maniera intensiva indici statistici per monitorare le prestazioni degli atleti durante il corso della stagione. Ne esistono a centinaia e vengono costantemente consultati da allenatori, giornalisti ed osservatori in quanto, talvolta, si rivelano indicatori del valore reale di un giocatore e descrittori di vari aspetti delle partite.
\par
L’NBA, lega professionistica della regione statunitense, oltre ad essere il più grande spettacolo sportivo al mondo è anche un investimento economico di notevole portata e con ampi margini di profitto. Nel 2017 il valore medio di una squadra ha raggiunto gli 1,1 miliardi di dollari e, per mantenere una continua crescita ed ambire al primo posto in classifica, ogni team cerca di impossessarsi sia dei migliori \textit{players} sul mercato sia degli strumenti più all’avanguardia per massimizzare le probabilità di posizionarsi tra i primi posti.
\par
È in quest’ultimo ambito che il Machine Learning può fornire il suo contributo: analizzando le varie performance e gli esiti delle partite passate si possono fare previsioni di vario genere su molti aspetti di questo sport. 
Questo è possibile perchè alcuni servizi e siti, ufficiali e non, mettono a disposizione le statistiche e i dati di cui abbiamo parlato prima in grandi datasets, che noi utilizzeremo come training sets.
\par
Strumenti di previsione e di classificazione possono essere utili anche per chi scommette sugli esiti delle partite, ambito anch’esso estremamente remunerativo e popolare negli Stati Uniti.
\par
Il nostro progetto si pone quindi l'obiettivo di individuare ed utilizzare un metodo di Machine Learning per predire il risultato (\textit{made} o \textit{missed}) di un tiro a canestro in NBA, utilizzando alcuni di tali indici ritenuti più rilevanti e significativi, rispetto alla totalità a disposizione.

\chapter{Dati}
\section{Acquisizione}

Dopo aver scelto il dominio applicativo di riferimento e aver deciso gli obiettivi principali, abbiamo scelto due dataset ospitati dalla piattaforma Kaggle.

\subsection{Shot\_logs.csv}

Questo dataset contiene 128 000 record riguardanti i tiri a canestro effettuati da 281 giocatori NBA diversi nella stagione 2014-2015. La fonte originale di questi dati è l’API pubblica del sito dell’NBA.
\begin{center}
	\begin{longtable}[m]{|m{8em} m{7em} m{16em}|} 

		\caption{Campi presi in considerazione del dataset \texttt{shot\_logs.csv}.\label{long}}\\

		\hline
		\bfseries{Attributo} & \bfseries{Tipo di dato} & \bfseries{Descrizione} \\
		\hline
		Location & factor(home, away) & In casa o fuori casa \\
		\hline
		W & factor(win, loss) & Partita vinta o persa dalla squadra del giocatore che ha fatto il tiro \\ 
		\hline
		Final margin & int & Scarto tra punteggi delle due squadre a fine partita \\ 
		\hline
		Shot number & int & Numero del tiro da inizio partita \\ 
		\hline
		Periodo & factor(1,2,3,4) & Periodo della partita, 4 nel basket \\ 
		\hline
		Game clock & Date/Time & Tempo della partita in cui si è effettuato il tiro \\ 
		\hline
		Shot Clock & Date/Time & Secondo in cui il tiro viene rilasciato dall’attaccante. L’intervallo va da 0 a 24, in quanto 24 è il tempo massimo a disposizione della squadra per effettuare un’azione \\ 
		\hline
		Dribbles & int & Numero di dribbling effettuati dall’attaccante prima del tiro \\ 
		\hline
		Touch Time & float & Tempo dal possesso palla in cui ha tirato il giocatore \\ 
		\hline
		Shot Dist & float & Distanza dal canestro al momento del tiro \\ 
		\hline
		PTS Type & factor(2,3) & Tipo di tiro effettuato (da 2 o da 3) \\ 
		\hline
		Shot Result & factor(made, missed) & L’obiettivo che vogliamo predire, esito del tiro (a segno o fallito) \\ 
		\hline
		Closest Defender & String & Nome del difensore più vicino all’attaccante \\ 
		\hline
		Close Def Dist & int & Distanza tra il difensore più vicino e l’attaccante \\ 
		\hline
		Player name & String & Nome dell’attaccante \\ 
		\hline
	\end{longtable}
\end{center}

\texttt{percentage\_previous\_game} è una nuova feature che abbiamo computato dai dati esistenti. Rappresenta la percentuale di successo al tiro fino a quel momento in stagione. Il suo valore è stato ottenuto il seguente script Python che, facendo uso della libreria \texttt{Pandas}, aggrega i dati delle partite precedenti a quella a cui si riferisce il tiro in questione.

\begin{code}
\captionof{listing}{datasets/shot\_logs\_nv.py}
\inputminted[breaklines]{python}{../datasets/shot_logs_nv.py}
\end{code}

Nel caso in cui la partita sia la prima della stagione, \texttt{percentage\_previous\_game} assume il valore \textit{TS}, ovvero la percentuale di successo al tiro riferito a tutta la sua carriera (\textit{True Shooting}), proveniente dall’altro dataset che abbiamo preso in considerazione, \texttt{Season\_stats.csv}.

\subsection{Season\_stats.csv}

Questo dataset contiene le statistiche inerenti agli atleti e alle loro performance, dal 1950 al 2017. I dati sono originari del sito Basketball Reference, prelevati usando tecniche di \textit{web scraping}.

\begin{center}
	\begin{longtable}[m]{|m{5em} m{7em} m{16em}|} 

		\caption{Campi presi in considerazione del dataset \texttt{Season\_stats.csv}.\label{long}}\\
		\hline
		\bfseries{Attributo} & \bfseries{Tipo di dato} & \bfseries{Descrizione} \\
		\hline

		Year & int & Anno di gioco\\ 
		\hline
		Player & String & Nome del giocatore\\ 
		\hline
		Pos & String & Ruolo nella squadra\\ 
		\hline
		Age & int & Età durante quella stagione\\ 
		\hline
		Games & int & Partite giocate in quella stagione\\ 
		\hline
		MP & int & Minuti totali giocati\\ 
		\hline
		PER & float & \textit{Player Efficiency Rating}, indicatore tecnico per valutare la bravura di un giocatore \\ 
		\hline
		TS\% & float & \textit{True Shooting Percentage}, un misuratore tecnico per valutare l’efficienza di tiro di un giocatore  \\ 
		\hline
		BLK\% & float & \textit{Block Percentage}, la percentuale di blocchi effettuati \\ 
		\hline

	\end{longtable}
\end{center}

% TODO descrivere PER e TS\%
% TODO forse questo va in Data Integration?

È stata effettuata un’integrazione affinchè ogni tiro registrato in \texttt{shot\_logs.csv} avesse anche le statistiche e gli indicatori tecnici dell’attaccante (colui che effettua il tiro) e del difensore (colui che attua il blocco per ostacolare il tiro). Dopodichè abbiamo scelto di rendere anonime le istanze del nuovo dataset rimuovendo nomi di attaccanti e difensori, cosicchè gli algoritmi di Machine Learning considerino e prevedano i risultati indipendentemente dai giocatori coinvolti, basandosi sulle loro statistiche, sul contesto di gioco e sugli altri valori forniti.

\subsection{Ipotesi fatte a priori}

Come per ogni altro processo statistico abbiamo innanzitutto dovuto individuare alcune ipotesi ritenute valide, assumendole tali, per poter orientare la parte di Machine Learning verso l’obbiettivo prefissato.
\par
Come fondamentale assunzione, riteniamo che l’esito del tiro a canestro possa essere dedotto in maniera abbastanza regolare dagli  attributi messi a disposizione dai due dataset utilizzati. In realtà, per ogni tentato canestro giocano un’innumerevole serie di fattori fisici ed ambientali: per esempio, l’esito viene anche influenzato da come il giocatore poggia il peso poco prima di tirare.
\par
Lavorando su performance dei giocatori dell’NBA, supponiamo anche che le loro abilità e la loro coerenza di tiro rimangano più o meno costanti (ad eccezione di poche partite sopra o sotto il loro standard, situazioni di \textit{outlier}).
\par
Ricordiamo inoltre che stiamo considerando la minoranza più forte in assoluto nel vasto mondo del basket, top 1\% dei giocatori agonistici. 
\par
%TODO controllare valore?
\par
Quindi, i risultati e le considerazioni che otterremo con il modello costruito, non saranno rilevanti per contesti diversi da quello considerato.
\par
I due dataset forniscono valori a noi utili che risalgono alla stagione di campionato 2014-2015. Volendo utilizzare i risultati per prevedere gli esiti dei tiri NBA della corrente stagione (2018-2019) dobbiamo supporre che non siano stati introdotti grandi cambiamenti nelle strategie e nel modo di allenare i giocatori, affinchè le dinamiche di gioco rimangano ancora molto simili.
\par
Inoltre, risulterà infattibile prevedere l’esito di tiri effettuati da nuovi giocatori entrati nella lega, poichè alcune delle metriche che utilizziamo sono relative allo storico delle loro performance. 
Una possibile soluzione a questo problema è ricavare valori approssimativi per questi indici consultando altre basi di dati che contengono dati relativi alle performance del giocatore in altre leghe.

\subsection{Misure di qualità dei dataset}

% 2 dimensioni di qualità dei dataset singoli
\subsection{Season\_Stats.csv}
In \texttt{Season\_Stats.csv} vengono definiti 53 attributi per ogni giocatore. Alcuni di questi tuttavia rappresentano delle criticità che sporcano il dataset, abbassandone la qualità sia a livello di schema sia a livello di istanze.
\par
Innanzitutto riscontriamo una mancata correttezza rispetto al modello: gli attributi \texttt{blanl} e \texttt{blank2} sono indubbiamente campi inutili: nessuna istanza ha valori definiti per queste due colonne. Sono probabilmente refusi della fase di scraping dei dati e vanno eliminati perchè non esistenti nel modello ER e perchè intaccano la minimality del dataset.
\par
Molte ennuple hanno valori mancanti soprattutto in alcuni attributi specifici, mostrando un chiaro caso di incompletezza: sul totale di 1283984 valori 154921 sono vuoti, per un’incompletezza che si attesta sullo 0,12\%.

\begin{center}
	\begin{longtable}[m]{|m{3em} m{3em} m{3em} m{3em} m{3em} m{3em} m{3em}|} 

		\caption{Incompletezza di \texttt{Season\_Stats.csv}.\label{long}}\\
		\hline
		\bfseries{Year} & \bfseries{Player} & \bfseries{Pos} & \bfseries{Age} & \bfseries{PER} & \bfseries{TS\%} & \bfseries{BLK\%} \\ 
		\hline
		0,003 & 0,003 & 0,003 & 0,003 & 0,024 & 0,006 & 0,158 \\
		\hline
	\end{longtable}
\end{center}

Infine la chiave principale per le istanze è un id incrementale che va da 0 a 24692. Si tratta di una chiave non molto utile per il tipo di dataset: le performance dei giocatori andrebbero indicizzati con un id univoco che si riferisce al singolo atleta. In questo modo sarebbe stato più facile eseguire l’integrazione tra i due dataset e, se avessimo voluto fare un’integrazione più ampia tra dataset di leghe diverse, sarebbe stato più facile rintracciare le statistiche dello stesso giocatore in campionati diversi.

\subsection{shot\_logs.csv}

\texttt{shot\_logs.csv}, presenta meno criticità, risultando più completo. Nonostante ciò, l’attributo \texttt{shot\_clock} presenta un’incompletezza dello 4,35\%: per 5577 istanze su 128069 il suo valore non è specificato.
Infine i valori delle colonne \texttt{player\_name} e \texttt{closest\_def} mancano di correttezza: giocatori con lo stesso id (quindi corrispondenti allo stesso giocatore) sono rappresentati con formati e capitalizzazioni diverse: \texttt{player\_name} infatti il segue il formato \texttt{Cognome, Nome} con le iniziali maiuscole, mentre in \texttt{closest\_def} il formato è \texttt{nome cognome} tutto in minuscolo.
\par
Questo problema caratterizza il 100\% delle istanze.

\chapter{Preparazione dati}

Abbiamo fatto notare che nel file \texttt{shot\_logs.csv} i nomi degli attaccanti (attributo \texttt{player\_name}) e i nomi dei difensori (attributo \texttt{closest\_defender}) hanno un formato diverso. Per eseguire una corretta integrazione con l’altro dataset è necessario quindi uniformare questi valori, permettendone l’identificazione come stesso giocatore, sia esso l’attaccante o il difensore nel tiro in questione. 
\par
Non si tratta di deduplication perchè quest’ultima mira ad individuare istanze che si riferiscono ad uno stesso concetto all’interno dello stesso database, mentre qui non abbiamo istanze duplicate, ma valori diversi che in domini differenti si riferiscono alla stessa entità.
\par
Inizialmente avevamo optato per una funzione che separasse i nomi in \textit{token}, li ordinasse lessicograficamente e che li risolvesse rilevando un match per una totale distanza di edit minore di 3 o maggiore solo in caso di diminutivi (per esempio: \texttt{J.J. Barea} e \texttt{Barea Jose Juan}).

Questa funzione è stata implementata in Python, nel file \texttt{datasets/match.py}.
\par
Il metodo funzionava correttamente, ma ci siamo poi accorti che due attributi del dataset (\texttt{player\_id} e \texttt{closest\_def\_id}) corrispondevano univocamente. Abbiamo quindi utilizzato la componente tMap in Talend per uniformare i nomi di attaccanti e difensori: usando tecniche di mapping abbiamo ricercato giocatori con lo stesso id e abbiamo sostituito il loro nome nella colonna \texttt{player\_name} (nome dell’attaccante) con quello della colonna \texttt{closest\_defender}.
\par
Il risultato di tale operazione ci ha permesso infatti una più agevole integrazione con l’altro dataset, poichè quest’ultimo non possiede un id univoco per i giocatori e il processo integrativo deve essere necessariamente effettuato facendo corrispondere i nomi degli atleti.
Infine bisogna specificare che per ovviare all’assenza di 5577 occorrenze nel campo ‘shot\_clock’, abbiamo deciso di utilizzare la tecnica di risoluzione dei conflitti meet in the middle, prendendo il valore medio di ciascun giocatore per il suddetto campo.
\par
Per i nostri scopi, dal dataset \texttt{Seasons\_Stats.csv} otteniamo un sottoinsieme delle statistiche dei giocatori riferite all’anno 2015. Al suo interno è possibile trovare fino a 4 duplicati della coppia di attributi Year-Player, in quanto nella lega è permesso giocare in 4 differenti squadre ogni stagione. Ciò è stato risolto aggregando i valori degli attributi sopra citati, gestendo le statistiche numeriche (\texttt{BLK}, \texttt{TS}, \texttt{PER}) considerando la media stagionale e considerando invece l’occorrenza più recente per gli attributi \texttt{Pos} ed \texttt{Age}.
\par
Alcuni record che riguardavano giocatori di cui non abbiamo informazioni per la stagione interessata, sono stati rimossi. E.g. \textit{Atila Dos Santos} compare in alcune istanze del primo dataset ma lo stesso non è presente nel secondo dataset che descrive le caratteristiche dei giocatori.

\section{Integrazione}

L’analisi esplorativa dei due datasets, anche in vista dell’applicazione dei metodi di Machine Learning, ha prodotto come risultato due nuovi datasets composti da un insieme ridotto di attributi, ritenuti rilevanti e non ridondanti per lo studio e gli obiettivi prefissati. Il numero di record è rimasto inalterato.
\par
L’integrazione dei due nuovi datasets ha avuto luogo in due passi separati. Inizialmente, sono stati utilizzati con una componente \textit{tMap} per associare al dataset dei tiri le statistiche degli attaccanti. Dopodiché, questo dataset consistente dei nuovi attributi è stato utilizzato assieme al dataset delle statistiche per associare, con un altro componente tMap, le informazioni relative al difensore.
\par
Non utilizzando un id univoco che identifica un giocatore per effettuare l’integrazione tra i due datasets ma i nomi stessi degli atleti, abbiamo optato per l’utilizzo della stessa funzione descritta nel paragrafo \textit{Preparazione dati}, così da uniformare i nomi presenti (scritti in minuscolo ordinati lessicograficamente) nei due datasets. 
\par
Effettuata l’integrazione, per verificare la consistenza del matching ad ogni passo è stato creato un file denominato \textit{rejected} in cui sono state inserite le istanze respinte, non trovando corrispondenza perfetta.
\par
Circa una decina di nomi sono risultati in questa lista. Questi casi sono stati risolti manualmente nella routine \texttt{EditShotLogs\_0.1.java}, eseguita in pipeline da Talend, poichè sarebbe stato problematico e dispendioso usare funzioni di edit distance appositamente parametrizzate per così pochi e particolari casi

\begin{code}
\captionof{listing}{Metodo matchNames da EditShotLogs\_0.1.java}
\begin{minted}{java}
public static String matchNames(String name1) {
	String[] def = {"Barea, Jose Juan", "Hardaway Jr., Tim", 
		"Aminu, Al-Farouq", "Nene", "Mbah a Moute, Luc",
		"Hayes, Charles", "Lucas III, John", "Taylor, Jeff",
		"Rice Jr., Glen", "Datome, Gigi", "McAdoo, James Michael"};
	String[] stats = {"J.J. Barea", "Tim Hardaway", 
		"Al-Farouq Aminu", "Nene Hilario", "Luc Mbah",
		"Chuck Hayes", "John Lucas", "Jeffery Taylor",
		"Glen Rice", "Luigi Datome", "James Michael"};
	String[] results = {"barea jj", "hardaway tim", 
		"al-farouq aminu", "nene hilario", "luc mbah",
		"chuck hayes", "john lucas", "jeffery taylor",
		"glen rice", "datome luigi", "james mcadoo michael"};
	if (Arrays.asList(def).contains(name1) || 
		Arrays.asList(stats).contains(name1)) {
		for (int i = 0; i < def.length; i++) {
			if (name1.equals(def[i]) || 
				name1.equals(stats[i])) {
				return results[i];
			}
		}  
	}
	String[] names1 = name1.replaceAll("[^a-zA-Z ]", "")
		.toLowerCase().split("\\s+");
	Arrays.sort(names1);
	String nuova = "";
	for (int i = 0; i < names1.length; i++) {
		nuova += names1[i];
		nuova += " ";
	}
	nuova = nuova.trim();
	return nuova;
}
\end{minted}
\end{code}

% todo aggiungere esempi di mismatch così si capisce che non aveva senso fare la funzione

\subsection{Misure di qualità del dataset integrato}

\chapter{Machine Learning}
\chapter{Conclusioni}
\end{document}