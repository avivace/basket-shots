\section{Implementazione di SVM}

%% forse questo pezzo andrebbe meglio altrove/conclusioni?
Nonostante gli aspetti positivi, è difficile determinare se la SVM sia la tecnica ideale per allenare il nostro dataset. Non abbiamo individuato un fattore discriminante che porti a preferirla rispetto ad altri modelli come le reti neurali oppure gli alberi di decisione.
%%
Inizialmente abbiamo provato ad implementare SVM con il package R \texttt{e1071}, ma il dataset utilizzato contiene un numero eccessivo di istanze e la computazione risultava troppo onerosa (\texttt{cannot allocate vector in R of size xx Gb}). Abbiamo sperimentato anche con il package \texttt{liquidSVM} ma non erano presenti di default funzionalità utili allo sviluppo del progetto come il calcolo della \textit{ROC} e \textit{AUC}, quindi abbiamo optato per il package \texttt{rminer} che implementa l’algoritmo della SVM di Kernlabs, basato sul lavoro di John Platt \cite{Platt99probabilisticoutputs} in cui viene descritto il metodo ad oggi più efficiente per ottenere stime probabilistiche sulla classificazione del test set con una SVM.

Dividiamo l'intero dataset in training e test set con questa funzione R:

\begin{minted}{r}
# Split dataset helper function
split.data = function(data, p = 0.8, s = 1) {
  set.seed(s)
  index = sample(1:dim(data)[1])
  train = data[index[1:floor(dim(data)[1] * p)], ]
  test = data[index[((ceiling(dim(data)[1] * p)) + 1):dim(data)[1]], ]
  return(list(train=train, test=test))
}
\end{minted}

La seguente riga produce e allena il modello:

\begin{minted}{r}
model=fit(shot_result~., trainset, model="ksvm", task="prob", 
          kernel="rbfdot", C=1)
\end{minted}

Ricordiamo che SVM un valore $ f(z) $ per ogni istanza di test $z$

\begin{align}
f(\mathbf{z}) &= \mathbf{w}^T\phi(\mathbf{z}) + b, \\
&= \sum_{i\in SV} \alpha_i y_i \kappa(\mathbf{x}_i,\mathbf{z}) + b.
\end{align}

Il risultato è un numero reale 
$$f(\cdot):\mathbb{R}^d \mapsto \mathbb{R}$$

Per ottenere etichette binarie utilizzando la soglia predefinita, viene applicata la funzione scalino (o Heaviside) su quest'ultimo:

$$\hat{y}(z) = \Theta(f(z))$$

Per ottenere $f(z)$ e non direttamente $\Theta(f(z))$, imposteremo il modello per eseguire il task \textit{prob} invece del predefinito \textit{class}, anche se il nostro è un problema di classificazione.

In questo modo, $f(z)$ da numero reale viene scalato in un intervallo $[0, 1]$ con la tecnica chiamata \textit{Platt scaling} \cite{Platt99probabilisticoutputs}, producendo quindi una stima della probabilità di classificazione.
\par
Aggiungendo alla funzione fit il parametro \texttt{search="heuristic10"} consentiamo la ricerca semiautomatica degli iperparametri ottimali su 10 range diversi per massimizzare la predizione del modello.
\par
Dopo una ricerca euristica su un sample di 25000 istanze, le metriche più accurate sono state ottenute con il valore C della SVM uguale a 1 e kernel \textit{RBFDOT} ossia Radial Basis Function Kernel.
\par
La funzione di discriminazione di tale Kernel è:
\begin{equation}
y\left(\mathbf{x}\right) = \sum_{i=1}^N w_i \, \phi\left(\left\|\mathbf{x} - \mathbf{x}_i\right\|\right)\label{RBFK}
\end{equation}
