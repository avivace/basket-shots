\section{Dominio di riferimento e obbiettivi}
Il settore sportivo professionistico è in costante ricerca di strumenti, modelli e modalità che permettano di analizzare gli andamenti delle partite e le performance dei singoli giocatori.

Il basket americano, ad esempio, utilizza in maniera intensiva indici statistici per monitorare le prestazioni degli atleti durante il corso della stagione. Ne esistono a centinaia e vengono costantemente consultati da allenatori, giornalisti ed osservatori in quanto, talvolta, si rivelano indicatori del valore reale di un giocatore e descrittori di vari aspetti delle partite.
\par
L’NBA, lega professionistica della regione statunitense, oltre ad essere il più grande spettacolo sportivo al mondo è anche un investimento economico di notevole portata e con ampi margini di profitto. Nel 2017 il valore medio di una squadra ha raggiunto gli 1,1 miliardi di dollari e, per mantenere una continua crescita ed ambire al primo posto in classifica, ogni team cerca di impossessarsi sia dei migliori \textit{players} sul mercato sia degli strumenti più all’avanguardia per massimizzare le probabilità di posizionarsi tra i primi posti.
\par
È in quest’ultimo ambito che il Machine Learning può fornire il suo contributo: analizzando le varie performance e gli esiti delle partite passate si possono fare previsioni di vario genere su molti aspetti di questo sport. 
Questo è possibile perchè alcuni servizi e siti, ufficiali e non, mettono a disposizione le statistiche e i dati di cui abbiamo parlato prima in grandi datasets, che noi utilizzeremo come training sets.
\par
Strumenti di previsione e di classificazione possono essere utili anche per chi scommette sugli esiti delle partite, ambito anch’esso estremamente remunerativo e popolare negli Stati Uniti.
\par
Il nostro progetto si pone quindi l'obiettivo di individuare ed utilizzare un metodo di Machine Learning per predire il risultato (\textit{made} o \textit{missed}) di un tiro a canestro in NBA, utilizzando alcuni di tali indici ritenuti più rilevanti e significativi, rispetto alla totalità a disposizione.
