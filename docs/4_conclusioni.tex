L’intero progetto, dalla scelta dei dataset all’analisi dei risultati dell’apprendimento automatico, si è rivelato essere molto ampio, coinvolgendo non solo i temi più comuni riguardo la Data Technology e il Machine Learning ma anche finezze statistiche, matematiche e informatiche. Ciò è sicuramente una dimostrazione di quante aree di studio possano essere coinvolte in questi due settori fortemente in crescita, sia nell’ambito della ricerca scientifica che nelle applicazioni di business.
I risultati del nostro modello, per quanto corretti, possiedono ancora ampio margine di miglioramento: il limite principale tuttavia non sono i limiti delle tecniche di apprendimento automatico e neppure la qualità delle istanze dei dataset coinvolti; è bensì la mancanza di alcuni attributi che conferirebbero al computatore una migliore rappresentazione della realtà di un tiro a canestro nell’NBA.
\section{Metriche mancanti}
Le metriche mostrano che la nostra SVM predice l’esito del tiro in maniera poco più che sufficiente ed è importante notare che predice meglio un tiro “missed” da uno “made” (come dimostrano i valori delle due recall).
Come dimostrato dai nostri colleghi di Stanford, tecniche di Machine Learning più avanzate come il Boosting possono massimizzare le metriche, sebbene rimangano ad una distanza minima dai nostri risultati.
\section{Sviluppi futuri}
Per ottenere predizioni sicure per il problema che abbiamo analizzato sarebbe necessario teoricamente implementare sensori come misuratori di inerzia, di rotazione e accelerometri sui giocatori e/o sulla palla da basket. Bisognerebbe anche registrare fattori come l’umidità dell’aria che potrebbero influenzare soprattutto i tiri più lunghi. Verrebbe quindi aggiunta una mole di dati continua da gestire con tecnologie IoT adeguate. Il basso livello di fattibilità riguardo a queste ultime considerazioni - difficilmente i giocatori di NBA giocherebbero con sensori addosso - ci permette di intravedere una soglia di incertezza sull’esito dei tiri che, quantomeno nel prossimo futuro, nessuno potrà superare.