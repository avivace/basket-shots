L'intero progetto, dalla scelta dei dataset all'analisi dei risultati dell'apprendimento automatico, si è rivelato essere molto ampio, coinvolgendo non solo i temi più comuni riguardo la Data Technology e il Machine Learning ma anche finezze statistiche, matematiche e informatiche. Ciò è sicuramente una dimostrazione di quante aree di studio possano essere coinvolte in questi due settori fortemente in crescita, sia nell'ambito della ricerca scientifica che nelle applicazioni di business.
I risultati del nostro modello, per quanto corretti, possiedono ancora ampio margine di miglioramento: il limite principale tuttavia non sono i limiti delle tecniche di apprendimento automatico e neppure la qualità delle istanze dei dataset coinvolti; è bensì la mancanza di alcuni attributi che conferirebbero al computatore una migliore rappresentazione della realtà di un tiro a canestro nell'NBA.

\section{Metriche mancanti}
\label{metrichemancanti}
A posteriori, l'insieme degli attributi usati per il modello è sembrato non sufficiente per garantire un livello di predizione affidabile.
Alcuni sviluppi futuri riguardano la considerazione di un più ampio insieme di attributi a discapito di altri.
Attualmente per stimare l'effettivo stato fisico e mentale del giocatore durante la stagione utilizziamo l'attributo percentage\_prev\_games. Supponendo di conoscere e poter utilizzare le statistiche delle stagioni precedenti, un indice ampiamente utilizzato per valutare il tiro di un giocatore è EFG (Effective Field Goal): consiste in una semplice statistica che classifica la bravura del giocatore separando i tiri secondo il loro valore potenziale (tiri da 2 e tiri da 3).
Questa metrica era la più avanzata nell'era precedente al \textit{visual tracking}: come già introdotto in 3.1, ad oggi sempre più fattori vengono analizzati per giudicare le prestazioni degli atleti, in qualsiasi ambito.
L'idea è quella di utilizzare una metrica basata su QSQ (Quantified Shot Quality) (linkare paper): essa misura quanto bene un giocatore tira in relazione alla difficoltà del tiro stesso e, a differenza di EFG, non si limita ad una statistica pura.
Ogni giocatore e tiro corrispondente potrebbe avere quindi come attributo un valore che rappresenta la difficoltà del tiro tenendo in considerazioni i diversi fattori sopra elencati.

Dall'analisi effettuata è un dato di fatto che il modello attuale non faccia discriminazioni profonde: sia con SVM che con DT l'importance è sbilanciata in favore dell'attributo \texttt{shot\_dist}. 
Questa interpretazione dei dati è però limitata in quanto non sempre essere vicini al canestro è sinonimo di efficacia: ad esempio può esserci una situazione in cui il difensore copre bene la zona e l'attaccante è sbilanciato. Attualmente il nostro modello non è in grado di riconoscere queste situazioni. Per lo stesso motivo, essere lontani dal canestro non è sinonimo di fallimento: ad esempio può esserci una situazione che rende agevole il canestro.
In questo senso l'attributo \texttt{closest\_def\_dist} aiuta il modello e infatti è secondo per contributo informativo.

\section{Riepilogo}

% To do
\par
Non essendo completamente soddisfatti dal livello di accuracy ottenuto, abbiamo cercato lavori analoghi al nostro per confrontare e verificare la possibile presenza di errori nella realizzazione del modello. \cite{predictingNBAst} fa un percorso analogo al nostro partendo dallo stesso datasets \textit{Shot logs}, confrontando più algoritmi e raggiungendo un'accuratezza  68\% utilizzando \textit{XGBoost}, un'implementazione di \textit{Boosting}. È un approccio che coinvolge diversi classificatori \textit{weak learners} e infine, ne combina le predizioni, pesate differentemente, con un ulteriore algoritmo di apprendimento artificiale (\textit{strong learner}).

Le metriche mostrano che la nostra SVM predice l'esito del tiro in maniera poco più che sufficiente ed è importante notare che predice meglio un tiro “missed” da uno “made” (come dimostrano i valori delle due recall).
Come dimostrato dai nostri colleghi di Stanford, tecniche di Machine Learning più avanzate come il Boosting possono massimizzare le metriche, sebbene rimangano ad una distanza minima dai nostri risultati.

\section{Sviluppi futuri}
Per ottenere predizioni sicure per il problema che abbiamo analizzato sarebbe necessario teoricamente implementare sensori come misuratori di inerzia, di rotazione e accelerometri sui giocatori e/o sulla palla da basket. Bisognerebbe anche registrare fattori come l'umidità dell'aria che potrebbero influenzare soprattutto i tiri più lunghi. Verrebbe quindi aggiunta una mole di dati continua da gestire con tecnologie IoT adeguate. Il basso livello di fattibilità riguardo a queste ultime considerazioni - difficilmente i giocatori di NBA giocherebbero con sensori addosso - ci permette di intravedere una soglia di incertezza sull'esito dei tiri che, quantomeno nel prossimo futuro, nessuno potrà superare.

