\section{Modelli di Machine Learning utilizzati}


% Scelta di C (complexity paramater) e Sigma (gaussian kernel parameter)

La scelta del modello di machine learning da utilizzare è determinata dalle caratteristiche dei dati a disposizione.

Il nostro dataset integrato ha un buon numero di attributi, sia numerici che categorici.
Abbiamo deciso di utilizzare SVM che è un algoritmo di apprendimento automatico supervisionato ampiamente utilizzato per problemi di classificazione. Il suo punto di forza è l'utilizzo del cosiddetto \textit{kernel trick}, strumento matematico per mappare l'input in uno spazio multi-dimensionale. SVM performa bene con dataset composti da tanti attributi e numerose osservazioni, adeguato sotto questo punto di vista al nostro caso.
La presenza di attributi categorici porterebbe a preferire gli alberi di decisione, ma esistono alcune tecniche che possono trasformare queste tipologie di campi affinchè siano compatibili anche con le SVM.

\subsection{Preparazione dei dataset per SVM}
\par
È stata stata applicata una normalizzazione del dataset per gli attributi numerici. È un procedimento che viene applicato frequentemente perché gli attributi sono calcolati con unità di misura differenti. In questo modo si evita che un attributo abbia un peso maggiore di un altro. È stato utilizzato per ciascun attributo il metodo
$$ \text{(min max)}\quad x = \dfrac{x - min(x)}{max(x) - min(x)} $$ 
molto utilizzato in letteratura in alternativa a
$$\text{(z score)}\quad x = \dfrac{x - \mu}{\sigma} $$
 in quanto gli attributi sono limitati in un range. Con $min max$ i valori di ciascun attributo vengono posti tra 0 e 1.

\par
Non essendo SVM una tecnica che gestisce attributi categorici, è stata applicata una tecnica di \textit{one-hot encoding} per gestirli. In questo modo vengono creati tanti nuovi attributi quante sono le categorie di ciascun attributo. Il nuovo attributo è 1 se l’attributo originale aveva quel determinato valore, 0 altrimenti.

\subsection{Implementazione di SVM}

Nonostante gli aspetti positivi, è difficile determinare se la SVM sia la tecnica ideale per allenare il nostro dataset. Non esiste infatti un motivo davvero discriminante che porti a preferirla rispetto ad altri modelli come le reti neurali oppure gli alberi di decisione.
Inizialmente abbiamo provato ad implementare la SVM con il package R chiamato e1071, ma il dataset utilizzato contiene un numero eccessivo di istanze e la computazione risultava troppo onerosa (cannot allocate vector in R of size xx Gb). Abbiamo sperimentato anche con il package liquidSVM ma non erano presenti di default funzionalità utili allo sviluppo del progetto come la ROC e l'AUC, quindi abbiamo optato per il package rminer che implementa l’algoritmo della SVM di Kernlabs, basato sul paper di Platt (LINKARE PAPER) in cui viene descritto il metodo ad oggi più efficiente per ottenere stime probabilistiche sulla classificazione del test set con una SVM.
Una volta importato e suddiviso il dataset per l’apprendimento automatico, la seguente riga produce e allena il modello:

\texttt{}

Sebbene il nostro sia un problema di classificazione, settare il parametro task a \textit{prob} ci permette di ottenere dalla logica di SVM il valore $ f(z) $ calcolato per ogni istanza di test $z$, piuttosto che il semplice segno aritmetico $sign(f(z))$ che indica una posizione minore o maggiore rispetto alla threshold, determinando la classe di appartenenza:

% f(z)=wTϕ(z)+b,=∑i∈SVαiyiκ(xi,z)+b 

%(https://stats.stackexchange.com/questions/134156/can-i-use-svm-classification-probability-for-ranking)


Con task uguale ”prob” il valore $f(z)$ viene automaticamente scalato in un range $[0, 1]$ con la tecnica chiamata \textit{Platt scaling}, producendo quindi una stima della probabilità di classificazione.
Aggiungendo alla funzione fit il seguente parametro \texttt{search="heuristic10"}

consentiamo la ricerca semiautomatica degli iperparametri ottimali su 10 range diversi per massimizzare la predizione del modello.

Dopo una ricerca euristica su un sample di 25000 istanze, le metriche più accurate sono state ottenute con il valore C della SVM uguale 1 e kernel “RBFDOT” ossia Radial Basis Function Kernel.
La funzione di discriminazione di tale Kernel è:
Formula di 

%https://it.wikipedia.org/wiki/Funzione_radiale_di_base

