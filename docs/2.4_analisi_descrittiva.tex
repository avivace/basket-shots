\pagebreak
\section{Analisi descrittiva del dataset integrato}

Dato il tipo di informazione che desideriamo ottenere dal dataset integrato, abbiamo valutato una serie di statistiche che caratterizzano fortemente i dati, ottenuti con lo script \texttt{count}.

\begin{table}[h!]
\centering
	\begin{tabular}{c c} 
	 
		Media di tiri in stagione& 455.72 \\
		Media di blocchi& 271.31 \\
		Media dei tiri con esito positivo& 206.05\\
		Media dei tiri con esito negativo& 249.67\\
		\end{tabular}
		\caption{Statistiche \textit{per giocatore} del dataset integrato}
\end{table}

La relazione tra numero di giocatori e tiri \textit{made} è mostrata nella \autoref{made_shots_hist}, mentre quella con i tiri \textit{missed} nella \autoref{missed_shots_hist}.

\begin{figure}
\caption{Frequenza di tiri con esito positivo}
\label{made_shots_hist}
\includegraphics[width=\linewidth]{made_shots_hist}
\end{figure}

\begin{figure}
\caption{Frequenza di tiri con esito negativo}
\label{missed_shots_hist}
\includegraphics[width=\linewidth]{missed_shots_hist}
\end{figure}
