\chapter{Acquisizione Dati}

Dopo aver scelto il dominio applicativo di riferimento e aver deciso gli obiettivi principali, abbiamo scelto due dataset ospitati dalla piattaforma Kaggle.

\section{Shot\_logs.csv}

Questo dataset contiene 128 000 record riguardanti i tiri a canestro effettuati da 281 giocatori NBA diversi nella stagione 2014-2015. La fonte originale di questi dati è l’API pubblica del sito dell’NBA.
\begin{center}
	\begin{longtable}[m]{|m{8em} m{7em} m{16em}|} 

		\caption{Campi presi in considerazione del dataset \texttt{shot\_logs.csv}.\label{long}}\\

		\hline
		\bfseries{Attributo} & \bfseries{Tipo di dato} & \bfseries{Descrizione} \\
		\hline
		Location & factor(home, away) & In casa o fuori casa \\
		\hline
		W & factor(win, loss) & Partita vinta o persa dalla squadra del giocatore che ha fatto il tiro \\ 
		\hline
		Final margin & int & Scarto tra punteggi delle due squadre a fine partita \\ 
		\hline
		Shot number & int & Numero del tiro da inizio partita \\ 
		\hline
		Periodo & factor(1,2,3,4) & Periodo della partita, 4 nel basket \\ 
		\hline
		Game clock & Date/Time & Tempo della partita in cui si è effettuato il tiro \\ 
		\hline
		Shot Clock & Date/Time & Secondo in cui il tiro viene rilasciato dall’attaccante. L’intervallo va da 0 a 24, in quanto 24 è il tempo massimo a disposizione della squadra per effettuare un’azione \\ 
		\hline
		Dribbles & int & Numero di dribbling effettuati dall’attaccante prima del tiro \\ 
		\hline
		Touch Time & float & Tempo dal possesso palla in cui ha tirato il giocatore \\ 
		\hline
		Shot Dist & float & Distanza dal canestro al momento del tiro \\ 
		\hline
		PTS Type & factor(2,3) & Tipo di tiro effettuato (da 2 o da 3) \\ 
		\hline
		Shot Result & factor(made, missed) & L’obiettivo che vogliamo predire, ossia se il tiro sia andato a segno o meno \\ 
		\hline
		Closest Defender & String & Nome del difensore più vicino all’attaccante \\ 
		\hline
		Close Def Dist & int & Distanza tra il difensore più vicino e l’attaccante \\ 
		\hline
		Player name & String & Nome dell’attaccante \\ 
		\hline
	\end{longtable}
\end{center}

\texttt{percentage\_previous\_game} è una nuova feature che abbiamo computato dai dati esistenti. Rappresenta la percentuale di successo al tiro fino a quel momento in stagione. Il suo valore è stato ottenuto il seguente script Python (datasets/shot\_logs\_nv.py) che, facendo uso della libreria \texttt{Pandas}, aggrega i dati delle partite precedenti a quella a cui si riferisce il tiro in questione.

\inputminted[breaklines]{python}{../datasets/shot_logs_nv.py}

Nel caso in cui la partita sia la prima della stagione, \texttt{percentage\_previous\_game} assume il valore \textit{TS}, ovvero la percentuale di successo al tiro riferito a tutta la sua carriera (\textit{True Shooting}), proveniente dall’altro dataset che abbiamo preso in considerazione, \texttt{Season\_stats.csv}.

\section{Season\_stats.csv}

Questo dataset possiede le statistiche inerenti agli atleti e alle loro performance, dal 1950 al 2017. I dati sono originari del sito Basketball Reference, prelevati usando tecniche di \textit{web scraping}.

\begin{center}
	\begin{longtable}[m]{|m{5em} m{7em} m{16em}|} 

		\caption{Campi presi in considerazione del dataset \texttt{Season\_stats.csv}.\label{long}}\\
		\hline
		\bfseries{Attributo} & \bfseries{Tipo di dato} & \bfseries{Descrizione} \\
		\hline

		Year & int & Anno di gioco\\ 
		\hline
		Player & String & Nome del giocatore\\ 
		\hline
		Pos & String & Ruolo nella squadra\\ 
		\hline
		Age & int & Età durante quella stagione\\ 
		\hline
		Games & int & Partite giocate in quella stagione\\ 
		\hline
		MP & int & Minuti totali giocati\\ 
		\hline
		PER & float & Player Efficiency Rating, indicatore tecnico per valutare la bravura di un giocatore \\ 
		\hline
		TS\% & float & True Shooting Percentage, un misuratore tecnico per valutare l’efficienza di tiro di un giocatore  \\ 
		\hline
		BLK\% & float & Block Percentage, la percentuale di blocchi effettuati \\ 
		\hline

	\end{longtable}
\end{center}

% TODO descrivere PER e TS\%
% TODO forse questo va in Data Integration?

È stata effettuata un’integrazione affinchè ogni tiro registrato in \texttt{shot\_logs.csv} avesse anche le statistiche e gli indicatori tecnici dell’attaccante (colui che effettua il tiro) e del difensore (colui che attua il blocco per ostacolare il tiro). Dopodichè abbiamo scelto di rendere anonime le istanze del nuovo dataset rimuovendo nomi di attaccanti e difensori, cosicchè gli algoritmi di Machine Learning considerino e prevedano i risultati indipendentemente dai giocatori coinvolti, basandosi sulle loro statistiche, sul contesto di gioco e sugli altri valori forniti.
