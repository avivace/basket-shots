\section{Acquisizione}

Dopo aver scelto il dominio applicativo di riferimento e aver deciso gli obiettivi principali, abbiamo scelto due dataset ospitati dalla piattaforma Kaggle.

\subsection{Shot logs}


\texttt{Shot\_logs.csv}\cite{shot_logs} contiene 128 000 record riguardanti i tiri a canestro effettuati da 281 giocatori NBA diversi nella stagione 2014-2015. La fonte originale di questi dati è l’API pubblica del sito dell’NBA.
\begin{center}
	\begin{longtable}[m]{|m{8em} m{7em} m{16em}|} 

		\caption{Campi presi in considerazione del dataset \texttt{shot\_logs.csv}.\label{long}}\\

		\hline
		\bfseries{Attributo} & \bfseries{Tipo di dato} & \bfseries{Descrizione} \\
		\hline
		Location & factor(home, away) & In casa o fuori casa \\
		\hline
		W & factor(win, loss) & Partita vinta o persa dalla squadra del giocatore che ha fatto il tiro \\ 
		\hline
		Final margin & int & Scarto tra punteggi delle due squadre a fine partita \\ 
		\hline
		Shot number & int & Numero del tiro da inizio partita \\ 
		\hline
		Periodo & factor(1,2,3,4) & Periodo della partita, 4 nel basket \\ 
		\hline
		Game clock & Date/Time & Tempo della partita in cui si è effettuato il tiro \\ 
		\hline
		Shot Clock & Date/Time & Secondo in cui il tiro viene rilasciato dall’attaccante. L’intervallo va da 0 a 24, in quanto 24 è il tempo massimo a disposizione della squadra per effettuare un’azione \\ 
		\hline
		Dribbles & int & Numero di dribbling effettuati dall’attaccante prima del tiro \\ 
		\hline
		Touch Time & float & Tempo dal possesso palla in cui ha tirato il giocatore \\ 
		\hline
		Shot Dist & float & Distanza dal canestro al momento del tiro \\ 
		\hline
		PTS Type & factor(2,3) & Tipo di tiro effettuato (da 2 o da 3) \\ 
		\hline
		Shot Result & factor(made, missed) & L’obiettivo che vogliamo predire, esito del tiro (a segno o fallito) \\ 
		\hline
		Closest Defender & String & Nome del difensore più vicino all’attaccante \\ 
		\hline
		Close Def Dist & int & Distanza tra il difensore più vicino e l’attaccante \\ 
		\hline
		Player name & String & Nome dell’attaccante \\ 
		\hline
	\end{longtable}
\end{center}

\texttt{percentage\_previous\_game} è una nuova feature che abbiamo computato dai dati esistenti. Rappresenta la percentuale di successo al tiro fino a quel momento in stagione. Il suo valore è stato ottenuto il seguente script Python che, facendo uso della libreria \texttt{Pandas}, aggrega i dati delle partite precedenti a quella a cui si riferisce il tiro in questione.

\begin{code}
\captionof{listing}{datasets/shot\_logs\_nv.py}
\inputminted[breaklines]{python}{../datasets/shot_logs_nv.py}
\end{code}

Nel caso in cui la partita sia la prima della stagione, \texttt{percentage\_previous\_game} assume il valore \textit{TS}, ovvero la percentuale di successo al tiro riferito a tutta la sua carriera (\textit{True Shooting}), proveniente dall’altro dataset che abbiamo preso in considerazione, \texttt{Season\_stats.csv}.

\subsection{Season stats}

\texttt{Season\_stats.csv} \cite{season_stats} contiene le statistiche inerenti agli atleti e alle loro performance, dal 1950 al 2017. I dati sono originari del sito Basketball Reference, prelevati usando tecniche di \textit{web scraping}.

\begin{center}
	\begin{longtable}[m]{|m{5em} m{7em} m{16em}|} 

		\caption{Campi presi in considerazione del dataset \texttt{Season\_stats.csv}.\label{long}}\\
		\hline
		\bfseries{Attributo} & \bfseries{Tipo di dato} & \bfseries{Descrizione} \\
		\hline

		Year & int & Anno di gioco\\ 
		\hline
		Player & String & Nome del giocatore\\ 
		\hline
		Pos & String & Ruolo nella squadra\\ 
		\hline
		Age & int & Età durante quella stagione\\ 
		\hline
		Games & int & Partite giocate in quella stagione\\ 
		\hline
		MP & int & Minuti totali giocati\\ 
		\hline
		PER & float & \textit{Player Efficiency Rating}, indicatore tecnico per valutare la bravura di un giocatore \\ 
		\hline
		TS\% & float & \textit{True Shooting Percentage}, un misuratore tecnico per valutare l’efficienza di tiro di un giocatore  \\ 
		\hline
		BLK\% & float & \textit{Block Percentage}, la percentuale di blocchi effettuati \\ 
		\hline

	\end{longtable}
\end{center}

% TODO descrivere PER e TS\%
% TODO forse questo va in Data Integration?

È stata effettuata un’integrazione affinchè ogni tiro registrato in \texttt{shot\_logs.csv} avesse anche le statistiche e gli indicatori tecnici dell’attaccante (colui che effettua il tiro) e del difensore (colui che attua il blocco per ostacolare il tiro). Dopodichè abbiamo scelto di rendere anonime le istanze del nuovo dataset rimuovendo nomi di attaccanti e difensori, cosicchè gli algoritmi di Machine Learning considerino e prevedano i risultati indipendentemente dai giocatori coinvolti, basandosi sulle loro statistiche, sul contesto di gioco e sugli altri valori forniti.

\subsection{Ipotesi fatte a priori}

Come per ogni altro processo statistico abbiamo innanzitutto dovuto individuare alcune ipotesi ritenute valide, assumendole tali, per poter orientare la parte di Machine Learning verso l’obbiettivo prefissato.
\par
Come fondamentale assunzione, riteniamo che l’esito del tiro a canestro possa essere dedotto in maniera abbastanza regolare dagli  attributi messi a disposizione dai due dataset utilizzati. In realtà, per ogni tentato canestro giocano un’innumerevole serie di fattori fisici ed ambientali: per esempio, l’esito viene anche influenzato da come il giocatore poggia il peso poco prima di tirare.
\par
Lavorando su performance dei giocatori dell’NBA, supponiamo anche che le loro abilità e la loro coerenza di tiro rimangano più o meno costanti (ad eccezione di poche partite sopra o sotto il loro standard, situazioni di \textit{outlier}).
\par
Ricordiamo inoltre che stiamo considerando la minoranza più forte in assoluto nel vasto mondo del basket, top 1\% dei giocatori agonistici. 
\par
%TODO controllare valore?
\par
Quindi, i risultati e le considerazioni che otterremo con il modello costruito, non saranno rilevanti per contesti diversi da quello considerato.
\par
I due dataset forniscono valori a noi utili che risalgono alla stagione di campionato 2014-2015. Volendo utilizzare i risultati per prevedere gli esiti dei tiri NBA della corrente stagione (2018-2019) dobbiamo supporre che non siano stati introdotti grandi cambiamenti nelle strategie e nel modo di allenare i giocatori, affinchè le dinamiche di gioco rimangano ancora molto simili.
\par
Inoltre, risulterà infattibile prevedere l’esito di tiri effettuati da nuovi giocatori entrati nella lega, poichè alcune delle metriche che utilizziamo sono relative allo storico delle loro performance. 
Una possibile soluzione a questo problema è ricavare valori approssimativi per questi indici consultando altre basi di dati che contengono dati relativi alle performance del giocatore in altre leghe.

\subsection{Misure di qualità dei dataset}

% 2 dimensioni di qualità dei dataset singoli
\subsubsection{Season stats}

Vengono definiti 53 attributi per ogni giocatore. Alcuni di questi tuttavia rappresentano delle criticità che sporcano il dataset, abbassandone la qualità sia a livello di schema sia a livello di istanze.
\par
Innanzitutto riscontriamo una mancata correttezza rispetto al modello: gli attributi \texttt{blanl} e \texttt{blank2} sono indubbiamente campi inutili: nessuna istanza ha valori definiti per queste due colonne. Sono probabilmente refusi della fase di scraping dei dati e vanno eliminati perchè non esistenti nel modello ER e perchè intaccano la minimality del dataset.
\par
Molte ennuple hanno valori mancanti soprattutto in alcuni attributi specifici, mostrando un chiaro caso di incompletezza: sul totale di 1283984 valori 154921 sono vuoti, per un’incompletezza che si attesta sullo 0,12\%.

\begin{center}
	\begin{longtable}[m]{|m{3em} m{3em} m{3em} m{3em} m{3em} m{3em} m{3em}|} 

		\caption{Incompletezza di \texttt{Season\_Stats.csv}.\label{long}}\\
		\hline
		\bfseries{Year} & \bfseries{Player} & \bfseries{Pos} & \bfseries{Age} & \bfseries{PER} & \bfseries{TS\%} & \bfseries{BLK\%} \\ 
		\hline
		0,003 & 0,003 & 0,003 & 0,003 & 0,024 & 0,006 & 0,158 \\
		\hline
	\end{longtable}
\end{center}

Infine la chiave principale per le istanze è un id incrementale che va da 0 a 24692. Si tratta di una chiave non molto utile per il tipo di dataset: le performance dei giocatori andrebbero indicizzati con un id univoco che si riferisce al singolo atleta. In questo modo sarebbe stato più facile eseguire l’integrazione tra i due dataset e, se avessimo voluto fare un’integrazione più ampia tra dataset di leghe diverse, sarebbe stato più facile rintracciare le statistiche dello stesso giocatore in campionati diversi.

\subsubsection{Shot logs}

Questo dataset presenta meno criticità, risultando più completo. Nonostante ciò, l’attributo \texttt{shot\_clock} presenta un’incompletezza dello 4,35\%: per 5577 istanze su 128069 il suo valore non è specificato.
Infine i valori delle colonne \texttt{player\_name} e \texttt{closest\_def} mancano di correttezza: giocatori con lo stesso id (quindi corrispondenti allo stesso giocatore) sono rappresentati con formati e capitalizzazioni diverse: \texttt{player\_name} infatti il segue il formato \texttt{Cognome, Nome} con le iniziali maiuscole, mentre in \texttt{closest\_def} il formato è \texttt{nome cognome} tutto in minuscolo.
\par
Questo problema caratterizza il 100\% delle istanze.
